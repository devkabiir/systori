\documentclass[9pt, oneside]{extletter}
\usepackage[utf8x]{inputenc}

\usepackage[defaultsans]{droidsans}
\renewcommand*\familydefault{\sfdefault} %% Only if the base font of the document is to be typewriter style
\usepackage[T1]{fontenc}
% droidsans font

% general page setup
\usepackage[a4paper,right=10mm, top=10mm,bottom=10mm, left=10mm]{geometry}
\usepackage{fancyhdr}%custom header package
\pagestyle{fancy}%custom header
\fancyhf{}%clear all
\renewcommand{\headrulewidth}{0pt}%clear head line

\usepackage{multicol}
\setlength\columnsep{0.4cm}

\usepackage{microtype}

\begin{document}

\large
\begin{center}\textbf{Allgemeine Geschäftsbedingungen}\end{center}

\begin{multicols}{3}
\fontsize{7.6}{8.6}
\selectfont
\textbf{I. Geltungsbereich}\\

1. Die allgemeinen Geschäftsbedingungen gelten für alle, mit der Firma Softronic Haustechnik und Holzbau GmbH abgeschlossenen Verträge, und zwar auch dann, wenn der Vertragspartner eigene Geschäftsbedingungen vorschreibt und wir diesen nicht ausdrücklich widersprechen.\\
2. Mündliche Vereinbarungen oder Nebenabreden vor Vertragsabschluss werden nur wirksam, wenn sie schriftlich getroffen werden.\\
\\
\textbf{II. Umfang der Lieferung oder Leistung}\\
\\
1. Für den Umfang der Lieferungen und Leistungen ist die getroffene Vereinbarung maßgebend. Nicht zum Lieferumfang gehören, sofern nicht anders vereinbart, Zu- und Verbindungsleistungen jeglicher Art, zu und zwischen den Lieferteilen, ihr mechanischer Aufbau, die Wandbefestigung sowie Versorgungs- und Kraftübertragungsmittel jeglicher Art. Schutzvorrichtungen werden insoweit mitgeliefert als dies gesetzlich vorgeschrieben oder ausdrücklich vereinbart ist.\\
2. Unsere Angebote erfolgen freibleibend hinsichtlich Preis, Menge, Lieferfrist und Liefermöglichkeit bis zu unserer schriftlichen Vertragsannahme.\\
3. Der Auftraggeber ist an seine Bestellung vier Wochen gebunden.\\
4. Die unseren Angeboten und Auftragsbestätigungen beigefügten Unterlagen wie Beschreibungen, Abbildungen, Zeichnungen, Maß- und Gewichtsangaben sind nur annähernd maßgebend, soweit sie nicht ausdrücklich als verbindlich bestätigt werden. An Kostenanschlagen, Zeichnungen und anderen Entwürfen behalten wir uns uneingeschränkt die Eigentums- und Urheberrechte vor. Diese Unterlagen dürfen nur nach unserer vorherigen Zustimmung Dritten zugänglich gemacht oder vom Auftraggeber außerhalb des Liefervertrages verwendet werden. Bei Erteilung des Auftrags geht das Eigentum an den genannten Unterlagen mit der vollständigen Bezahlung auf den Auftraggeber über. Dies gilt nicht für die Urheberrechte.\\
5. Unterlagen des Auftraggebers dürfen wir Dritten soweit zugänglich machen, wie diesen zulässigerweise ganz oder teilweise Lieferungen oder Leistungen übertragen worden sind.\\
6. Erforderliche behördliche oder sonstige Genehmigungen hat der Auftraggeber auf seine Kosten selbst zu beschaffen. Soweit hierfür von uns gefertigte Unterlagen vorgelegt werden müssen, wird insoweit die Einwilligung zur Vortage erteilt.\\
\\
\textbf{III. Preise}\\
\\
1. Die Preise gelten bei Lieferung ohne Aufstellung, Montage und/oder Inbetriebnahme. Wird von uns nur die Lieferung übernommen, ist in der vereinbarten Vergütung das Anliefern bis zur Baustelle enthalten, soweit diese mit einem Lkw erreichbar ist, widrigenfalls bis zum nächsten Punkt, der mit dem Lkw erreicht werden kann. Die vereinbarte Vergütung gilt nur bei der vertraglich vorgesehenen Lieferung sowie - soweit vereinbart - ununterbrochenen Montage mit anschließender Inbetriebnahme. Werden auf Verlangen des Auftraggebers Teillieferungen erbracht bzw. die Montage mit anschließender Inbetriebnahme unterbrochen, sind wir berechtigt, für die durch eine mehrfache Anlieferung (Montage) entstehenden Mehrkosten einen angemessenen Zuschlag zu erheben. Diese gilt für den Fall, dass sich bei der Durchführung des Auftrages nicht im Vertrag vorgesehene Arbeiten als notwendig erweisen oder auf Verlangen des Auftraggebers erbracht werden.\\
2. Sofern die Leistung nicht innerhalb 4 Monaten nach Vertragsschluss erbracht werden soll, sind wir berechtigt, die vereinbarte Vergütung aufgrund zwischenzeitlich eingetretener Lohn- und/oder Materialpreiserhöhung zu erhöhen. Die gilt auch für den Fall, dass die Leistung aus Gründen, die der Auftraggeber zu vertreten hat, erst nach Ablauf der 4-Monatsfrist erfolgen kann.\\
\\
\textbf{IV. Zahlungsbedingungen}\\
\\
1. Unsere Rechnungen sind, soweit nichts anderes vereinbart ist, ohne Abzug sofort fällig. Bei Überschreitung des Zahlungstermins sind wir berechtigt ohne Mahnung bankübliche Zinsen zu verlangen. Wechsel und Schecks werden nur zahlungshalber angenommen, die Zahlung gilt erst bei Gutschrift als erfolgt. Anfallende Kosten trägt der Auftraggeber.\\
2. Die Aufrechnung durch den Auftraggeber ist ausgeschlossen, soweit es sich nicht um unbestrittene oder rechtskräftig festgestellte Gegenforderungen handelt. Ohne unsere Zustimmung dürfen Ansprüche des Auftraggebers nicht abgetreten werden. Der Auftraggeber ist nicht berechtigt, einen Dritten zur Geltendmachung seiner Ansprüche uns gegenüber im eigenen Namen zu ermächtigen. Bei mehreren Auftraggebern gilt das Abtretungs- und Ermächtigungsverbot auch für Rechtsgeschäfte untereinander.\\
\\
\textbf{V. Eigentumsvorbehalt}\\
\\
Gelieferte Waren bleiben bis zur vollständigen Bezahlung des Kaufpreises unser Eigentum. Vor Bezahlung sind Verpfändungen oder Sicherungsübereignung untersagt, die Weiterveräußerung ist nur Wiederverkäufern im gewöhnlichen Geschäftsgang unter der Bedingung gestattet, dass der Wiederverkäufer von seinem Kunden Bezahlung erhält Der Besteller tritt jetzt schon seine Ansprüche aus der Wiederveräußerung von gelieferten Gegenständen zur Sicherheit an uns ab. Bei der Entgegennahme von Bargeld wird vereinbart, dass wir unmittelbar Eigentümer werden. Der Auftraggeber ist verpflichtet, den entgegengenommenen Geldbetrag gesondert für uns zu verwahren. Im Scheck - Wechselverkehr erlischt der Eigentumsvorbehalt erst bei Einlösung und Gutschrift auf unserem Konto.\\
\\
\textbf{VI. Frist für Lieferungen oder Leistungen}\\
\\
1. Hinsichtlich der Frist für Lieferungen oder Leistungen sind die beiderseitigen schriftlichen Erklärungen maßgebend. Die Einhaltung einer verbindlichen Frist setzt den rechtzeitigen Eingang sämtlicher vom Besteller zu liefernden Unterlagen, der erforderlichen Genehmigungen, Freigaben, der rechtzeitigen Klarstellung und Genehmigung der Pläne, die Einhaltung der vereinbarten Zahlungsbedingungen und sonstigen Verpflichtungen voraus. Werden diese Voraussetzungen nicht rechtzeitig erfüllt, so wird diese Frist angemessen verlängert.\\
2. Bei Lieferung ohne Aufstellung oder Montage gilt die Frist als eingehalten, wenn die betriebsbereite Sendung innerhalb der vereinbarten Liefer- oder Leistungsfrist zum Versand gebracht oder abgeholt worden ist. Verzögert sich die Ablieferung aus Gründen, die der Auftraggeber zu vertreten hat, so gilt die Frist als eingehalten mit der Meldung der Versandbereitschaft innerhalb der vereinbarten Frist. Die Frist gilt ferner als eingehalten bei Lieferung mit Aufstellung und Montage, sobald diese innerhalb der vereinbarte Frist erfolgt ist oder innerhalb der vereinbarten Frist begonnen und durch Umstände nicht ausgeführt werden kann, die der Auftraggeber zu vertreten hat.\\
3. Ist die Nichteinhaltung der Frist für Lieferungen oder Leistungen nachweislich auf höhere Gewalt, Streik, Aussperrung oder den Eintritt unvorhersehbarer Ereignisse zurückzuführen, so wird die Frist angemessen verlängert. In einem solchen Fall ist der Auftraggeber berechtigt, uns schriftlich eine angemessene Nachfrist zur Erbringung der Lieferungen und Leistungen zu setzen; nach fruchtlosem Ablauf der Nachfrist kann er vom Vertrag zurücktreten . Weitergehende Ansprüche sind ausgeschlossen.\\
4. Die Abnahme der Lieferung ist Hauptverpflichtung des Auftraggebers. Ist ein Termin für die Lieferungen oder Leistungen verbindlich vereinbart, kommt der Auftraggeber mit der Nichtabnahme in Verzug, ohne dass es einer Mahnung bedarf.\\
5. Bei Vereinbarung einer unverbindlichen Frist ist der Auftraggeber einen Monat nach Ablauf zur Abnahme verpflichtet. Wir sind berechtigt, von dem Auftraggeber die Entgegennahme der Lieferung oder Leistung innerhalb dieses Zeitraums zu verlangen. Ist die Lieferung oder Erbringung von Leistungen von einem Abruf des Auftraggebers abhängig, so hat dieser den Abruf innerhalb eines Jahres nach Vertragsschluss zu erteilen. Wir sind berechtigt vom Auftraggeber die Erteilung des Abrufs bis zu diesem Zeitpunkt zu verlangen.\\
6. Wird der Versand oder die Zustellung auf Wunsch des Auftraggebers verzögert, können wir, beginnend einen Monat nach Anzeige der Versandbereitschaft, Lagergeld in Hohe von 0,5 v. H. des offenen Rechnungsbetrages für jeden angefangenen Monat berechnen. Das Lagergeld wird auf 5 v. H. begrenzt, es sei denn, dass höhere Kosten nachgewiesen werden.\\
\\
\textbf{VII. Haftung und Mangel}\\
\\
1. Der Auftraggeber ist verpflichtet, uns unverzüglich von dem Auftreten von Mangeln zu unterrichten, aus welchen Gewährleistungsansprüche abgeleitet werden sollen. Bei Bestehen der Gewährleistungsansprüche steht uns zunächst das Recht zu, innerhalb einer angemessenen Frist die nach unserem Ermessen notwendig erscheinende Nachbesserung oder wahlweise Ersatzlieferungen vorzunehmen. Vor Ablauf dieser Frist ist der Auftraggeber nicht berechtigt, den Mangel selbst oder durch Dritte beseitigen zu lassen, es sei denn es ist Gefahr in Verzug. Verweigert der Auftraggeber uns die Möglichkeit, innerhalb einer angemessenen Frist die Ausbesserung oder wahlweise Ersatzlieferungen vorzunehmen, sind Wir von der Mangelhaftung befreit. Nach Fehlschlagen der Nachbesserung oder der Ersatzlieferung kann der Auftragnehmer Herabsetzung der Vergütung oder nach seiner Wahl Rückgängigmachung des Vertrages verlangen.\\
2. Eine Mangelhaftung für natürliche Abnutzung wird nicht übernommen, ferner nicht für Schaden die nach der Auslieferung in Folge fehlerhafter oder nachlässiger Behandlung oder Bedienung, übermäßiger Beanspruchung, ungeeigneter Betriebsmittel, oder solcher chemischer, elektrochemischer oder elektrischer Einflüsse und Umweltbedingungen entstehen, die nach dem Vertrag nicht vorausgesetzt sind. Werden seitens des Auftraggebers oder eines Dritten unsachgemäß vorgenommene Änderungen und Instandsetzungsarbeiten vorgenommen, wird die Haftung für die daraus entstehenden Folgen ausgeschlossen.\\
\\
\textbf{VIII. Sonstige Schadensersatzansprüche}\\
\\
Schadensersatzansprüche des Auftraggebers aus positiver Forderungsverletzung, aus der Verletzung von Pflichten bei den Vertragsverhandlungen und aus unerlaubter Handlung werden ausgeschlossen, soweit nicht auf Grund Vorsatz oder grober Fahrlässigkeit des Auftragnehmers, seiner gesetzlichen Vertreter oder seiner Erfüllungsgehilfen zwingend gehaftet wird. Für diese Fälle ist der Schadensersatzanspruch nicht ausgeschlossen.\\
\\
\textbf{IX. Kündigung, Schadensersatz wegen Nichterfüllung}\\
\\
1. Kündigt der Auftraggeber aus Gründen, die wir nicht zu vertreten haben, sind wir berechtigt, 35\% des Auftragsnettowertes der noch nicht erbrachten Lieferungen oder Leistungen als die vereinbarte Vergütung unter Abzug ersparter Aufwendungen zu verlangen. Dem Auftraggeber steht der Nachweis offen, dass wir durch die Kündigung höhere Aufwendungen erspart haben.\\
2. Steht uns ein Schadensersatzanspruch wegen Nichterfüllung gegenüber dem Auftraggeber zu, sind wir ebenfalls berechtigt, als Schaden 35\% des Auftragsnettowertes der noch nicht erbrachten Lieferungen und/oder Leistungen zu verlangen. Dem Auftraggeber steht der Nachweis offen, dass ein Schaden überhaupt nicht oder in geringerer Höhe entstanden ist.\\
\\
\textbf{X. Gerichtsstand}\\
\\
Gerichtsstand ist Mannheim.\\
\\
\textbf{XI. Salvatorische Klausel}\\
\\
Sollte einer der vorstehenden Bestimmungen unwirksam sein, berührt dies die Wirksamkeit des Vertrages im übrigen nicht.
\end{multicols}

\end{document}
